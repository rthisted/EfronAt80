% Abstract
In 1976, Efron and I published an article studying the problem of estimating the number of unseen species, using as a running example the (unseen) words in Shakespeare's vocabulary that he hadn't employed in the extant corpus of his work.  The approaches we considered required computations that were not then---and are not today---in the repertoire of standard statistical software.  Roughly half way between the appearance of our paper and today, Buckheit and Donoho (1995) introduced ideas of reproducible research.  Here we consider the reproducibility of the figures and tables in the Shakespeare paper, what we might have done in 1976 that would have aided reproducibility in 2018, and what lessons that exercise suggests for maximizing the likelihood that today's computationally intensive research can be reproduced in decades to come.
